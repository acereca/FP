\documentclass[11pt,a4paper]{article}
\usepackage[utf8]{inputenc}
\usepackage[german]{babel}
\usepackage[T1]{fontenc}
\usepackage{amsmath}
\usepackage{amsfonts}
\usepackage{amssymb}
\usepackage{graphicx}
\begin{document}
For the invesigation of the normal Zeeman effect we used a Cadmium lamp and placed it into a magnetic field. In the first part the splitting of the red spectral line was observed with a Lummer - Gehrcke plate. Furthermore, we determined the Bohr magneton $\mu_B = (9.956 \pm 0.414) \cdot 10^{-24} \; \frac{J}{T}$. For the second part a Czerny - Turner spectrometer was used to analyze the wavelength of the red line of the Cadmium spectrum to be $\lambda_{Cd} = (643.927 \pm 0.571) \; nm$. 
\end{document}