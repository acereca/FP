\section{Diskussion und Zusammenfassung}
  Der erste Teil des Versuchs bestand darin, zuerst herrauszufinden ob in unseren Beobachtungen der Hysterese Effekt zu berücksichtigen war und der Erhalt des Verhältnisses zwischen elektrischem Strom und Magnetfeldstärke.
  Bei uns war der Hysterese-Effekt zu vernachlässigen.

  Weiterhin wurden qualitative Aussagen über die Polarisation der Übergänge in Cadmium untersucht, welche mit den gemachten Beobachtungen übereinstimmen.

  Danach konnten die beobachteten Spektallinien einer Cd-Lampe mit Hilfe einer Lummer-Gehrcke Platte und einer CCD Kamera untersucht werden, welche uns aus der Beobachtung des Interferenzmusters von $\pi$- und $\sigma$-Linien einen Wert für das Bohr'sche Magneton, auf zwei Arten lieferte. Beide Methoden ergaben dabei unterschiedliche Ergebnisse, wobei das mit Hilfe des arithmetischen Mittels berechnete Erbegnis genauer mit dem Literaturwert überein stimmt. Mathematisch sollten bei beiden Methoden die selben Werte gefunden werden.
  \begin{align*}
    \mu_{B,1} &= \magnetonOne\\
    \mu_{B,2} &= \magnetonTwo\\
    \mu_{B,Theo} & = \magnetonTheo
  \end{align*}
  Dabei sollte erwähnt werden, dass der lineare Fit einen eher größeren Fehler besitzt, da dieser nur zwischen drei Messpunkten stattfand und möglicherweise kein lineares Verhalten aufweist, das kann daran liegen, dass das Magnetfeld im Bereich zwischen 12 und 13A bereits in Sättigung geht und somit unser Zusammenhang zwischen Strom und Magnetfeld nicht mehr als linear angenommen werden sollte. Was ausserdem in einer geringeren Aufspaltung der Linien resultiert.
  Man könnte diesen Effekt vermeiden, indem man kleinere Änderungen des Stroms betrachtet.
  Des Weiteren ist das Einbeziehen beider Fehler, in x- und y-Richtung, algorithmentechnisch sehr schwierig umzusetzen, weshalb wir uns hier nur auf den größeren Fehler beschränkt haben. Dadurch ist der ermittelte Fehler größer abzuschätzen.

  Im Zweiten Teil des Versuchs, galt es Wellenlängen den Linien des Cadmium-Spektrums zuzuordnen. Dabei fanden wir für die Hauptlinie eine sehr gute Übereinstimmung mit dem Literaturwert
  \begin{align*}
    \begin{rcases}
      \lambda_{Cd} &=\lambdaCd \quad\\
      \lambda_{Cd, Theo} &= \lambdaCdTheo
    \end{rcases}
    \quad\text{Abw.: }\SI{.14}{\sigma}
  \end{align*}

  Zudem sollte eine Wellenlänge einer unbekannten Linie zugeordnet werden, diese stimmt unserer Meinung nach mit einer Linie des Xe(I)-Spektrums überein

  \begin{align*}
    \begin{rcases}
      \lambda_{uk} &= \lambdaUk\quad\\
      \lambda_{Xe I, } &= \lambdaXe
    \end{rcases}
    \quad\text{Abw.: }\SI{.38}{\sigma}
  \end{align*}
