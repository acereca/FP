\thispagestyle{empty}
	\null\vspace{40mm}
	\begin{center}
		{
			\Large Zeemanspektroskopie
			\footnote{
				\noindent Versuch F44, ausgeführt am 26.6.17,
				Betreuer: Frans Schotsch,
				lange besondere Auswertung
			}
		}\\[15mm]
		Patrick Nisblé und David Bubeck

		\vspace{25mm}

		\parbox{0.9\textwidth}{
			Abstract:\\
			\small For the invesigation of the normal Zeeman effect we used a Cadmium lamp and placed it into a magnetic field. In the first part the splitting of the red spectral line was observed with a Lummer-Gehrcke plate. Furthermore, we determined the Bohr magneton $\mu_B = (9.956 \pm 0.414) \cdot 10^{-24} \; \frac{J}{T}$. For the second part a Czerny-Turner spectrometer was used to analyze the wavelength of the red line of the Cadmium spectrum to be $\lambda_{Cd} = (643.927 \pm 0.571) \; nm$. 
		}
	\end{center}

	\vfill
	Als besondere Auswertung testiert: Datum, Unterschrift:
	\vspace{20mm}
	%% Rueckseite des Titelblatts leer. Bei einseitigem Druck entfernen
	\newpage
	\null\thispagestyle{empty}
