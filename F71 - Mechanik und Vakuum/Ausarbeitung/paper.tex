\documentclass[12pt]{article}
\usepackage{graphicx}         % fuer das Einbinden von Grafiken
\usepackage[ngerman]{babel}   % weglassen, wenn in Englisch
%% wenn Sie das ngerman package benutzen, koennen Umlaute als "a.. geschrieben
%% werden, sonst \"a..
\usepackage[utf8]{inputenc}
%% dieses package erlaubt, bei deutscher Tastatur Umlaute, ß direkt einzugeben

\textwidth=170mm
\textheight=250mm
\hoffset= -20mm       % may need change
\voffset= -25mm       % may need change

%% everything after % is a comment in LATEX

\begin{document}
	
	\thispagestyle{empty}
	\null\vspace{40mm}
	\begin{center}
		{
			\Large  Die Nutzung von Vakuumpumpen für\\ unterschiedliche Anwendungen aufgrund ihrer Funktionsweise,\\ 
			sowie die Mechanik der Evakuierung	
			\footnote{
				\noindent Versuch F71, ausgeführt am 24.4.17,
				Betreuer: Frederik Arand,
				kurze besondere Auswertung
			}
		}\\[15mm]
		P. Nisblé und D. Bubeck
		
		\vspace{25mm}
		
		\parbox{0.9\textwidth}{
			Abstract:    
			\small The abstract should preferentially be in English. Here we explain in a
			few lines (i) what was done, and (ii) what the results were.
		}
	\end{center}
	
	\vfill
	Als besondere Auswertung testiert: Datum, Unterschrift:
	\vspace{20mm}
	
	%% Rueckseite des Titelblatts leer. Bei einseitigem Druck entfernen
	\newpage  
	\null\thispagestyle{empty} 
	
	%\newpage     % Inhaltsverzeichnis, koennte man bei langer Version machen
	%\tableofcontents 
	
	\newpage
	
	\pagenumbering{arabic} %% start page 1 
	\section{Einleitung}
		Diese Reihe von Versuchen dienen zur Orientierung und Nutzung von Apparaturen die Evakuierung benötigen, sowie zur Verständnis der Vakuumtechnik und auch deren Grenzen. In geringem Maße auch der Sensibilisierung für zuvor unbekannte Fehlerquellen die in der Vakuumtechnik zu Fehlern führen können.
	
	\section{Versuchsanordnung}
		
	
	\section{Versuchsdurchführung}
	
	
	\subsection{Eichung}
	
	
	\section{Ergebnisse}
	
	
	
	\section{Diskussion}
	
	Hier werden alle wesentlichen Ergebnisse nochmals angefuehrt und diskutiert. 
	
	Am Schluss kann man noch eine allgemeinere Bemerkung zum Versuch machen.
	
	
	\newpage 
	
	\begin{thebibliography}{00}   % {00}: max 2-stellig
		
		\bibitem{afo} F. Afo, Nature 15 (1905) 23
		\bibitem{uwe} Uwe Ludwig, private Mitteilung
		\bibitem{karl} Karl Popper, Phys.~Rev.~Lett.~95 (2001) 25
		\bibitem{dipl} K. Winter, Diplomarbeit Heidelberg (1968)
		\bibitem{bibel} Genesis 3,4
		
	\end{thebibliography}
	
\end{document}