\documentclass[12pt, a4paper]{scrartcl}

\usepackage{a4wide}
\usepackage{graphicx}      
\usepackage{pdflscape}
\usepackage{float}
\usepackage{amsmath}
\usepackage{siunitx}
\sisetup{seperr}
\sisetup{expproduct=\cdot,per=frac,fraction=nice}
\usepackage{afterpage}
\usepackage{microtype}
\usepackage[
		colorlinks=false,
		urlcolor=blue,
		linkcolor=white
]{hyperref}
\usepackage{booktabs}
\usepackage[font=footnotesize,labelfont=bf]{caption}

\usepackage[utf8]{inputenc}
%\usepackage[T1]{fontenc}
\usepackage[ngerman]{babel}

\renewcommand*\rmdefault{cmss}

\newcommand{\sciv}[4]{#1=#2\cdot10^{#3}\ #4}

\begin{document}
	
	\thispagestyle{empty}
	\null\vspace{40mm}
	\begin{center}
		{
			\Large Mechanik und Vakuum
			\footnote{
				\noindent Versuch F71, ausgeführt am 24.4.17,
				Betreuer: Frederik Arand,
				kurze besondere Auswertung
			}
		}\\[15mm]
		P. Nisblé und D. Bubeck
		
		\vspace{25mm}
		
		\parbox{0.9\textwidth}{
			Abstract:\\    
			\small We observe the use of vacuum pumps for different applications, as well as the mechanics behind pulling vacuums with those pumps. Our main focus is on rotary vein pumps and turbo molecular pumps (TMP).
            
            The content of this experiment aims at determining the capacity of suction of a TMP, and to find conductance and holes of an apparature.
		}
	\end{center}
	
	\vfill
	Als besondere (kurze) Auswertung testiert: Datum, Unterschrift:
	\vspace{20mm}
	%% Rueckseite des Titelblatts leer. Bei einseitigem Druck entfernen
	\newpage  
	\null\thispagestyle{empty} 
	
	%\newpage     % Inhaltsverzeichnis, koennte man bei langer Version machen
	%\tableofcontents 
	\begin{figure}[H]
		\centering
		\includegraphics[height=.6\paperheight]{foto1}
		\caption{Apparatur die für den Versuch genutzt wird}
	\end{figure}
	
	\newpage
	
	\pagenumbering{arabic} %% start page 1 
	\section{Einleitung}
		Diese Reihe von Versuchen dienen zur Orientierung und Nutzung von Apparaturen die Evakuierung benötigen, sowie zur Verständnis der Vakuumtechnik und auch deren Grenzen. In geringem Maße auch der Sensibilisierung für zuvor unbekannte Fehlerquellen die in der Vakuumtechnik zu Fehlern führen können.\\\\
		Der komplette Versuch ist getrennt in 6 Teilversuche: \cite{skript}
		\begin{enumerate}
			\item Funktionsweise einer Drehschieberpumpe
			
				Beobachtung einer Drehschieberpumpe in Betrieb und Bestimmung des maximal erreichbaren Vakuums (nach Abb. \ref{fig:anord1})

			\item Abpumpen kondensierbarer Dämpfe
				
				Beobachtung der selben Drehschieberpumpe unter Abpumpen kondensierbarer Dämpfe und dem daraus resultierenden maximalen Vakuum (nach Abb. \ref{fig:anord2})

			\item Funktionsweise von Molekular- und Turbomolekularpumpe (TMP)
			
                Beobachtung einer Hybridpumpe mit Turbo- und Gaedestufe in Betrieb
                (nach Abb. \ref{fig:anord3})
			
			\item Saugvermögen der TMP
			
            	Messung des Saugvermögens der TMP mithilfe einer Kapillaren
                (nach Abb. \ref{fig:anord4})
			
			\item Bestimmung des Leitwerts von Rohr und Blende
            	
                (nach Abb. \ref{fig:anord5})
			
			\item Lecksuche
			
            	mit Teslatransformator und Heliumlecksucher
                (nach Abb. \ref{fig:anord6})
		\end{enumerate}
	\newpage
	\section{Versuchsdurchführung}
	\subsection{Inbetriebnahme der Drehschieberpumpe}
	
        \begin{figure}[H]
            \centering
            \includegraphics[width=.5\paperwidth]{aufbau21}
            \caption{generalisierter Aufbau zur Beobachtung der Funktionsweise einer Drehschieberpumpe}
            \label{fig:anord1}
        \end{figure}
    
    	
    	
    	Es stellt sich ein Enddruck, der Drehschieberpumpe, bei \begin{align*}
            (1.2\pm 0.1)\cdot 10^{-1}\ mbar
        \end{align*}
        ein.
        Aufgrund des zu großen Enddrucks und der darauffolgenden Lecksuche, welche negativ war, wird angenommen, dass sich noch Wasser im System befindet. (Versuchsteil 2 wurde vorgezogen)
    
    
    
    \subsection{Abpumpen kondensierbarer Dämpfe}
    
		\begin{figure}[H]
			\centering
			\includegraphics[width=.5\paperwidth]{aufbau22}
			\caption{Vakuum-Blockschaltbild zum Versuch des Abpumpens kondensierbarer Dämpfe}
            \label{fig:anord2}
		\end{figure}
	
		Beobachtung:
		\begin{itemize}
			\item Wasser beginnt bei niedrigem Druck an zu sieden
			
			$\rightarrow$ Schalte Gasballast zu um Kondensation des Wasserdampfes zu verhindern			
			
			\item Wasser gefriert bei $\sim 5.8\ mbar$
			
		\end{itemize}

	
	\subsection{Inbetriebnahme einer Turbomolekularpumpe}
	
        \begin{figure}[H]
            \centering
            \includegraphics[width=.55\paperwidth]{aufbau23}
            \caption{Aufbau zur Beobachtung der Funktionsweise von Molekular- und Turbomolekularpumpe}
            \label{fig:anord3}
        \end{figure}
    
    	Die Turbomolekularpumpe wird nun nach \cite{skript}, wie in Aufbau \ref{fig:anord3} hinzugeschalten. Die Apparatur wurde eingeschaltet und über Nacht laufen gelassen.
    	
    	Über Nacht stellt sich der Druck im Rezipienten auf 
        $(1.1\pm0.1)\cdot 10^{-6}\ mbar$
        ein. (gemessen mit dem Dual-Gauge-Messgerät)
        
        Die Frequenz der Turbomolekularpumpe beträgt $f = 1.5\ kHz$
    
    
    \subsection{Saugvermögen der TMP}
    
        \begin{figure}[H]
            \centering
            \includegraphics[width=.55\paperwidth]{aufbau24}
            \caption{Bestimmung des Saugvermögens einer TMP}
            \label{fig:anord4}
        \end{figure}
    
        Zum Messen des Saugvermögens werden nun Glaskapillaren an die Vakuumkammer angeflanscht. Dann wird der gewünschte Druck eingestellt und ein Tropfen Seifenwasser in die Kapillare gegeben. Dieser wird beobachtet und das angesaugte Volumen pro Zeit notiert (s. Tab \ref{m11} und \ref{m12}).
    	
    	Die Kapillaren haben die folgenden Größen:
        \begin{itemize}
            \item $0.1\ ml$
            \item $0.2\ ml$
            \item $0.5\ ml$
            \item $1\ ml$
        \end{itemize}
    	
    	
    	Zuletzt wird ein Kolben angebracht:
        
        
        (Diese Messungen geschehen bei Äußerem Normaldruck)
        Anschließend wird langsam belüftet um die TMP nicht zu überhitzen
    	
    	\begin{itemize}
    		\item Gesamtvolumen: $(35\pm3)\ ml$ (Schätzung)
    		\item Masse: $m=44.67\ g$
    		\item Skalenteilung: $0.5\ ml$
            % 26.5mm/5ml
    		\item Durchmesser: $d=(15.5\pm0.1)\cdot10^{-3}\ m$
    	\end{itemize}
    
    \subsection{Leitwert von Rohr und Blende}
    
        \begin{figure}[H]
            \centering
            \includegraphics[width=.55\paperwidth]{aufbau24.png}
            \caption{Aufbau zur Bestimmung des Leitwerts von Rohr und Blende}
            \label{fig:anord5}
        \end{figure}
    
    	Der Rezipient wird nun leer gepumpt und anschließend mittels Binärventil der gewünschte Druck eingestellt. Bei verschiedenen Kombinationen aus Rohr und Blende in der Führung wird der Druck jeweils oben am Führrohr und unten am Rezipienten gemessen (mithilfe der Dual-Gauge)
    	
    	\begin{itemize}
            %TODO
    		\item Maße Blende:
    			\begin{align*}
    				d=&(4.2\pm 0.1)\cdot 10^{-3}\ m\\
    				h=&(5.0\pm 0.1)\cdot 10^{-3}\ m
    			\end{align*}
    			
    		\item Maße Rohr:
    			\begin{align*}
    				d=&(12.0\pm 0.1)\cdot 10^{-3}\ m\\
    				h=&(1.00\pm 0.05)\ m
    			\end{align*}
    			
    		\item Maße Halterohr:
    			\begin{align*}
    				d=&(52.0\pm 0.1)\cdot 10^{-3}\ m\\
    				h=&(1.00\pm 0.05)\ m
    			\end{align*}
    	\end{itemize}
    
    	
   	
   		Nach der Messung wird die Pumpe belüftet. Das Halterohr abgeschraubt und die Blende entfernt. Nach erneutem anbringen des Halterohrs wird eine weitere Druckmessung durchgeführt.
   	
   		Die Pumpe wird belüftet und als letzte Messreihe lediglich die Blende in das Halterohr montiert. Der Rezipient wird erneut evakuiert und eine erneute Druckmessung durchgeführt.
   		
   		(Messungen in Tab \ref{m21} bis Tab \ref{m23})
   	    
    \subsection{Lecksuche}
    
		Zunächst wird der Aufbau aus Versuchsteilen 4 \& 5 benutzt\\\\	
		$\rightarrow$ Über Nacht stellt sich ein Druck von 
        \SI{3.6+-0.1 e-1}{mbar}
        ein.\\\\
		Die Pumpen werden erneut eingeschaltet. Es wird ein aus Glas geschmolzenes Leck aus dem Auslass des Ventils am Rezipienten angebracht.\\
		Der Teslatransformator wird dann auf die Glaskapillare gehalten.\\
		$\rightarrow$ Der Luftstrom ist durch die Gasentladung sehr deutlich zu sehen.
        \begin{figure}[H]
            \centering
            \includegraphics[width=.6\paperwidth]{foto2}
            \caption{Lecksuche mit Teslatrafo}
        \end{figure}	
		Für die Gegenstromlecksuche wird folgender Aufbau verwendet:
		
		\begin{figure}[H]
			\centering
			\includegraphics[width=.3\paperwidth]{aufbau262}
			\caption{Prinzipschaltbild des im Versuchsteil 6 eingesetzten Gegenstromlecksuchers}
			\label{fig:anord6}
		\end{figure}
		Mittels einer Heliumgasflasche wird an verschiedenen Stellen Helium auf die Apparatur gegeben und der Ausschlag am Spektrometer beobachtet.\\
		$\rightarrow$ Es kann eine poröse Dichtung und ein Materialfehler im Bauteil ausfindig gemacht werden, die offensichtlich für das Leck im Aufbau verantwortlich sind.

	\section{Ergebnisse}
	\subsection{Inbetriebnahme der Drehschieberpumpe}
	
		Nach Inbetriebnahme der Drehschieberpumpe f/"allt der Druck innerhalb von
        $2\ min$
        auf 
        $(1.2\pm0.1)\cdot 10^{-1}\ mbar$
        und verweilt dort.
		
		Das Vakuum kann mit einer Drehschieberpumpe nicht beliebig gut werden, da die Drehschieberpumpe bei einem gewissen Druck das Gas nicht mehr genügend komprimieren kann, was zum Ausstoßen des Gases nötig ist.
		Es stellt sich aber ein Gleichgewicht bei einem bestimmten Druck ein.
        		
	\subsection{Abpumpen kondensierbarer Dämpfe}
		
		\begin{figure}[H]
			\centering
			\includegraphics[width=.5\paperwidth]{phasen-wasser}
			\caption{Phasendiagramm von Wasser (nach \cite{wikibooks})}
		\end{figure}
    
        Im zweiten Teil des Versuchs k\"onnen wir beobachten, wie Wasser verschiedene Zustände durchläuft:
        
        \begin{itemize}
            \item Das Wasser startet bei Normaldruck und Raumtemperatur
            
            \item Der Druck im Rezipienten fällt zu Beginn sehr stark, da die Pumpe eine große Gasmenge fördern kann. Die Temperatur des Wassers sollte dabei konstant bleiben. Ab einem Druck um $11\ mbar$ sollte das Wasser zu kochen beginnen.
            
            \item Durch das siedende Wasser und den entstehenden Wasserdampf kann die Drehschieberpumpe nicht weiter arbeiten\\
            $\rightarrow$ Das Gasballastventil wird geöffnet
            
            \item Die Pumpe kann nun weiter arbeiten und der Zustand des Wassers wandert entlang der Dampfdruckkurve
            
            \item bei etwa \SI{5.8\pm 0.1}{\milli \bar}, gefriert das Wasser, da nun der Tripelpunkt überschritten ist\\(Literaturwert: $6\ mbar\ \rightarrow\ 2\sigma$, da Kammer und Messger\"at \"uber einen Schlauch verbunden sind ist der Druck in der Kammer geringf\"ugig h\"oher)
        \end{itemize}

    \subsection{Inbetriebnahme einer TMP}
    
        Die TMP erreichte (nach einer Nacht) einen Druck von 
        $(1.1\pm 0.1)\cdot10^{-6}\ mbar$, 
        was den durch die Drehschieberpumpe erreichten Druck um 5
        Größenordnungen absenkt, was solange dauert da Gase im Rezipienten desorbieren (Pseudoleck)
        
    \subsection{Saugvermögen der TMP}
        
        Nach dem Anbringen der Kapillare stellt sich ein Gleichgewicht ein, das durch die Kapillare angesaugte Luftvolumen $V$ entspricht dann genau dem im Rezipienten abgesaugten Volumen $V_R$ (pro Zeit):
        \begin{align}
            pV=nRT
        \end{align}
        (Druck p, Volumen V, Stoffmenge n, Gaskonstante R und Temperatur T)
        \begin{align}
            \frac{p_{außen}V}{t} =& \frac{p_{innen}V_R}{t}\\
            \Leftrightarrow S=&\frac{V_R}{t}=\frac{V_R}{t}\cdot \frac{p_{außen}}{p_{innen}}
            %\Delta S=&\sqrt{
            %    (\frac{\Delta t}t)^2
            %    (\frac{\Delta V}V)^2
            %    (\frac{\Delta p_{innen}}{p_{innen}})^2
            %}\cdot S
        \end{align}
	
        Auf den Fehler von $p_{außen}$ wird verzichtet und es der Literaturwert von $1013\ hPa$ verwendet.
        Die errechneten Werte für das Saugvermögen sind in den Tabellen \ref{m11} und \ref{m12}
        zu finden, die Daten wurden zudem in Abbildung
        \ref{diagram:saugv}
        aufgetragen und ein linearer Fit, mit den letzten beiden Werten der Kapillarmessung und dem zweiten Wert der Kolbenmessung, durchgeführt.
        
        \begin{align*}
            \SI{8.28+-0.04}{\litre\per\second}
        \end{align*}
        
        \begin{itemize}
            \item Deutlich zu sehen ist, dass die mit dem Kolben gemessenen Werte systematisch zu hoch liegen. Dies liegt daran, dass der Kolben durch sein eigenes Gewicht noch zusätzlich Luft von außen herein drückt.
            
            \item Es war zudem ein recht poröser Schlauch am Kolben angebracht, der für ein größeres Leck sorgt und somit die Messung verfälscht.
            
            \item Fehler bei den Kapillaren können dadurch zustande gekommen sein, dass die Kapillare in ihrer Halterung leicht schräg angebracht ist und nicht waagerecht. Der Tropfen erf\"ahrt daher eine Hangabtriebskraft, weg vom Rezipienten
            
            \item nach der zweiten Messung wurde dem Wasser etwas mehr Seife hinzugefügt, um die Reibung des Tropfens in der Kapillare zu minimieren
            
            \item bei hohen Dr\"ucken f\"allt das Saugverm\"ogen ab, da die TMP hier nicht mehr mit molekularer Str\"omung arbeiten kann
        \end{itemize}
        Wie könnte man die Genauigkeit der Messung erhöhen?
        
        Zu Beginn bei sehr kleinem eingesaugtem/ Volumen könnten noch dünnere Kapillare die zeitliche Genauigkeit erhöhen, da der Tropfen dann mit höherer Geschwindigkeit durch die Kapillare laufen würde.
        
        \begin{landscape}
        	%\thispagestyle{empty}
        	\begin{figure}[h!]
        		%TODO: add fit and labels
        		\hspace*{-40mm}\includegraphics[width=1.8\textwidth]{../24-f1}
        		\caption{Saugvermögen der TMP inklusive linearem Fit}
        		\label{diagram:saugv}
        	\end{figure}
        	
        \end{landscape}
    
    
    \subsection{Leitwert von Rohr und Blende}
    
        Bei verschiedenen eingebauten Teilen (Rohr, Blende) werden die Druckdifferenzen oberhalb und unterhalb des Bauteils gemessen um den Leitwert des Aufbaus zu bestimmen. Mit dem in Teil 3.4 bestimmten Saugvermögen lässt sich über die Druckdifferenz $\Delta p$ auf den Leitwert L schließen, mittels:
        
        \begin{align}
            L=&\frac{Q}{\Delta p}
            \intertext{(Saugleistung Q)}
            \text{mit } Q=&S\cdot p_{unten}\\
            \Rightarrow L=&\frac{S\cdot p_{unten}}{\Delta p},\quad
            \Delta p= p_{oben} - p_{unten}
        \end{align}
        Die berechneten Leitwerte für die 3 Konfigurationen aus Rohr und Blende sind in Diagramm \ref{diagram:leitwert}
        aufgetragen. zu sehen ist bei allen 3 Konfigurationen ein Anstieg des Leitwertes hin zu kleinen Drücken
        ($p < 10^{-4}\ mbar$)
        und hin zu sehr großen Drücken.
        ($p > 2\cdot 10^{-3}\ mbar$)
        Dies hat damit zu tun, dass in dieser Berechnung das Saugvermögen über den gesamten Druckbereich als konstant angenommen wird. Wie in Diagramm
        \ref{diagram:saugv} zu sehen, ist dies keinesfalls so.
        Der Leitwert ist in diesen Bereichen wegen des zu groß angenommenen Saugvermögens selbst auch viel zu hoch. Bei den hohen Drücken gibt es außerdem den Effekt, dass hier schon der Übergang von der molekularen Strömung zur laminaren Strömung stattfindet und der Leitwert damit ansteigt.
        Zur Bestimmung des Leitwerts im molekularen Bereich wird daher ein Fit (konstant) an die Messpunkte im Bereich 
        $2\cdot 10^{-4}$ bis $2\cdot 10^{-3}$ mbar
        gemacht:
        
        \begin{align*}
            %TODO: val
            L_{Rohr}=&\SI{.659+-.032}{\litre\per\second}\\
            L_{Rohr und Blende}=&\SI{.524+-.029}{\litre\per\second}\\
            L_{Blende}=&\SI{2.13+-0.08}{\litre\per\second}
        \end{align*}
        
        \begin{landscape}
        	%\thispagestyle{empty}
        	\begin{figure}[h!]
        		\hspace*{-40mm}\includegraphics[width=1.8\textwidth]{../25-f1.png}
        		\caption{Leitwertbestimmung anhand des Diagramms}
        		\label{diagram:leitwert}
        	\end{figure}
        \end{landscape}
        
        \subsubsection*{Überprüfung der Kirchhoffschen Regeln}
        
        Serienschaltung:
        \begin{align}
			\frac 1{L_{R+B}} = \frac 1 {L_R} + \frac 1 {L_B} \quad \Leftrightarrow \quad L_{R+B} = \frac{L_R L_B}{L_R+L_B}
			\label{f:kirch}
        \end{align}
        
        Mit den experimentell bestimmten Werten für $L_R$ und $L_B$ folgt:
        \begin{align}
        	L_{R+B} = \SI{.503+-.019}{\litre\per\second}
        \end{align}
        
        (Abw: \SI{0.6}{\sigma}) Was mit unserem experimentellen Wert sehr gut übereinstimmt und die Kirchhoffschen Regeln als Analogie bestätigt.
        
        \subsubsection*{theoretische Leitwerte}
        
        \begin{align}
        	L_{\text{theo,lam,Rohr}}&=\frac{\pi}{8} r^4 \frac{\overline p}{\eta l}&\overline p = (p_o+p_u)/2, \eta= 
        	\SI{17.1}{\micro\pascal\per\second}\\
        	&= \SI{0.48+-0.03}{\litre\per\second} &\text{(letzte Drücke aus Tab \ref{m22})}\\
        	\Rightarrow \text{Abw.: } \SI{4}{\sigma}\\
        	\hline\nonumber\\
        	L_{\text{theo,mol,Rohr}}&=\frac 8 3 \frac{r^3}{l} \sqrt{\frac{\pi R T}{2 M}} &R=\SI{8.314}{\joule\per\mole\per\kelvin}, M_{Luft} = \SI{28.96}{\gram\per\mole}\\
        	&=\SI{.209+-.012}{\litre\per\second}\\
        	\Rightarrow \text{Abw.: } \SI{13.2}{\sigma}\\
        	\hline\nonumber\\
        	L_{\text{theo,mol,Blende}}
        	&=\SI{362}{\meter\per\second} r^2\\
        	&=\SI{}{\litre\per\second}\\
        	\Rightarrow \text{Abw.: } \SI{4.8}{\sigma}\\
        	\hline\nonumber\\
        	\Rightarrow L_{theo,R+B} 
        	&= \SI{.368+-.02}{\litre\per\second}
        	&(\text{nach }\ref{f:kirch})\\
        	\Rightarrow \text{Abw.: \SI{4.5}{\sigma}}&&\text{zum experimentellen Wert}
        \end{align}
	
	
	\newpage 
	
	\begin{thebibliography}{00}   % {00}: max 2-stellig
		
		\bibitem{skript} Versuchsskript zu Versuch F70
		
		\bibitem{wikibooks} de.wikibooks.org: Aggregatszustandsänderungen\\ (\url{https://de.wikibooks.org/wiki/Physik_in_unserem_Leben/_Aggregatzustands%C3%A4nderungen})
		
	\end{thebibliography}
	
    \newpage
    \section{Anhang}
    
    \begin{figure}[H]
        \centering
        \begin{tabular}{llll}
\toprule
{} &           p / mbar &       t / s &           V / ml \\
\midrule
0 &  (7.10+/-0.10)e-06 &  50.8+/-3.0 &  0.0100+/-0.0005 \\
\bottomrule
\end{tabular}

        \caption{Messreihe 1.1 - Saugvermögen bei Kapillaren}
        \label{m11}
    \end{figure}

   	\begin{figure}[H]
        \centering
        \begin{tabular}{lll}
\toprule
                                    p / mbar &                                    t / s &                                      V / ml \\
\midrule
 $\left(1.10 \pm 0.10\right) \times 10^{-3}$ &  $\left(5.45 \pm 0\right) \times 10^{1}$ &  $\left(5.00 \pm 0.50\right) \times 10^{0}$ \\
 $\left(2.90 \pm 0.10\right) \times 10^{-4}$ &  $\left(6.10 \pm 0\right) \times 10^{1}$ &  $\left(2.00 \pm 0.50\right) \times 10^{0}$ \\
 $\left(3.10 \pm 0.10\right) \times 10^{-3}$ &  $\left(2.40 \pm 0\right) \times 10^{1}$ &  $\left(5.50 \pm 0.50\right) \times 10^{0}$ \\
 $\left(1.00 \pm 0.10\right) \times 10^{-2}$ &  $\left(1.71 \pm 0\right) \times 10^{1}$ &  $\left(1.00 \pm 0.05\right) \times 10^{1}$ \\
 $\left(3.40 \pm 0.10\right) \times 10^{-2}$ &  $\left(1.13 \pm 0\right) \times 10^{1}$ &  $\left(1.50 \pm 0.05\right) \times 10^{1}$ \\
\bottomrule
\end{tabular}

        \caption{Messreihe 1.2 - Saugvermögen bei Kolben}
        \label{m12}
    \end{figure}

    \begin{figure}[H]
        \centering
        \begin{tabular}{ll}
\toprule
                                $p_o$ / mbar &                                 $p_u$ / mbar \\
\midrule
 $\left(5.80 \pm 0.10\right) \times 10^{-3}$ &  $\left(5.40 \pm 0.10\right) \times 10^{-5}$ \\
 $\left(2.40 \pm 0.10\right) \times 10^{-2}$ &  $\left(9.70 \pm 0.10\right) \times 10^{-5}$ \\
 $\left(4.40 \pm 0.10\right) \times 10^{-2}$ &  $\left(1.70 \pm 0.10\right) \times 10^{-4}$ \\
 $\left(7.10 \pm 0.10\right) \times 10^{-2}$ &  $\left(3.30 \pm 0.10\right) \times 10^{-4}$ \\
 $\left(9.50 \pm 0.10\right) \times 10^{-2}$ &  $\left(5.60 \pm 0.10\right) \times 10^{-4}$ \\
 $\left(1.30 \pm 0.10\right) \times 10^{-1}$ &  $\left(1.10 \pm 0.10\right) \times 10^{-3}$ \\
 $\left(1.80 \pm 0.10\right) \times 10^{-1}$ &  $\left(2.00 \pm 0.10\right) \times 10^{-3}$ \\
 $\left(2.20 \pm 0.10\right) \times 10^{-1}$ &  $\left(3.30 \pm 0.10\right) \times 10^{-3}$ \\
 $\left(2.60 \pm 0.10\right) \times 10^{-1}$ &  $\left(5.70 \pm 0.10\right) \times 10^{-3}$ \\
 $\left(3.20 \pm 0.10\right) \times 10^{-1}$ &  $\left(1.00 \pm 0.10\right) \times 10^{-2}$ \\
\bottomrule
\end{tabular}

        \caption{Messreihe 2.1: Rohr und Blende}
        \label{m21}
    \end{figure}

    \begin{figure}[H]
        \centering
        \begin{tabular}{ll}
\toprule
                                $p_o$ / mbar &                                 $p_u$ / mbar \\
\midrule
 $\left(5.90 \pm 0.10\right) \times 10^{-3}$ &  $\left(5.50 \pm 0.10\right) \times 10^{-5}$ \\
 $\left(2.20 \pm 0.10\right) \times 10^{-2}$ &  $\left(1.00 \pm 0.10\right) \times 10^{-4}$ \\
 $\left(4.20 \pm 0.10\right) \times 10^{-2}$ &  $\left(1.80 \pm 0.10\right) \times 10^{-4}$ \\
 $\left(6.60 \pm 0.10\right) \times 10^{-2}$ &  $\left(3.40 \pm 0.10\right) \times 10^{-4}$ \\
 $\left(8.70 \pm 0.10\right) \times 10^{-2}$ &  $\left(5.50 \pm 0.10\right) \times 10^{-4}$ \\
 $\left(1.20 \pm 0.10\right) \times 10^{-1}$ &  $\left(1.00 \pm 0.10\right) \times 10^{-3}$ \\
 $\left(1.50 \pm 0.10\right) \times 10^{-1}$ &  $\left(1.80 \pm 0.10\right) \times 10^{-3}$ \\
 $\left(2.00 \pm 0.10\right) \times 10^{-1}$ &  $\left(3.20 \pm 0.10\right) \times 10^{-3}$ \\
 $\left(2.40 \pm 0.10\right) \times 10^{-1}$ &  $\left(5.60 \pm 0.10\right) \times 10^{-3}$ \\
 $\left(3.10 \pm 0.10\right) \times 10^{-1}$ &  $\left(1.10 \pm 0.10\right) \times 10^{-2}$ \\
\bottomrule
\end{tabular}

        \caption{Messreihe 2.2: Rohr}
        \label{m22}
    \end{figure}

    \begin{figure}[H]
        \centering
        \begin{tabular}{ll}
\toprule
                                $p_o$ / mbar &                                 $p_u$ / mbar \\
\midrule
 $\left(9.40 \pm 0.10\right) \times 10^{-4}$ &  $\left(5.60 \pm 0.10\right) \times 10^{-5}$ \\
 $\left(3.20 \pm 0.10\right) \times 10^{-3}$ &  $\left(9.80 \pm 0.10\right) \times 10^{-5}$ \\
 $\left(7.30 \pm 0.10\right) \times 10^{-3}$ &  $\left(1.80 \pm 0.10\right) \times 10^{-4}$ \\
 $\left(1.40 \pm 0.10\right) \times 10^{-2}$ &  $\left(3.20 \pm 0.10\right) \times 10^{-4}$ \\
 $\left(2.40 \pm 0.10\right) \times 10^{-2}$ &  $\left(5.60 \pm 0.10\right) \times 10^{-4}$ \\
 $\left(4.10 \pm 0.10\right) \times 10^{-2}$ &  $\left(1.00 \pm 0.10\right) \times 10^{-3}$ \\
 $\left(6.20 \pm 0.10\right) \times 10^{-2}$ &  $\left(1.90 \pm 0.10\right) \times 10^{-3}$ \\
 $\left(9.10 \pm 0.10\right) \times 10^{-2}$ &  $\left(3.30 \pm 0.10\right) \times 10^{-3}$ \\
 $\left(1.20 \pm 0.10\right) \times 10^{-1}$ &  $\left(5.40 \pm 0.10\right) \times 10^{-3}$ \\
 $\left(1.70 \pm 0.10\right) \times 10^{-1}$ &  $\left(1.00 \pm 0.10\right) \times 10^{-2}$ \\
\bottomrule
\end{tabular}

        \caption{Messreihe 2.3: Blende}
        \label{m23}
    \end{figure} 



\end{document}