% !TeX root = ../pres.tex

\section{Einleitung}
    \subsection{Grundlagen}
        \begin{myframe}{\subsecname}
            Beschreibung des Zeeman - Effekt in einer Näherung:
            \begin{itemize}
                \item $\vec{l} = \vec{r} \times \vec{p} = m_e \cdot v \cdot \vec{n}$
                \item Elektron kann mit einem Strom $ I $ und magnetischen Moment $\mu_{l}$ beschrieben werden.
                \item Interaktion mit externen Magnetfeld und dem magnetischen Moment ergibt sich Änderung der potentiellen Energie
                \item mit $\vec{l}$ und $\vec{B} \parallel \vec{l}$ erhält man
                \begin{itemize}
                    \item[] $\Delta E_{pot} = \frac{e \cdot \hbar}{2m_e} \cdot m_l \cdot B = \mu_B \cdot m_l \cdot B $
                    \begin{itemize}
                        \item[] $\mu_{B}$ : Bohr'sche Magneton
                    \end{itemize}
                \end{itemize}
            \end{itemize}
        \end{myframe}

        \begin{myframe}{\subsecname}
            Atome mit mehreren Elektronen:
            \begin{itemize}
                \item $\vec{L} \vec{S}$ - Kopplungsnäherung
                \item Gesamtdrehimpuls $\vec{J} = \vec{L} + \vec{S}$
                \begin{itemize}
                    \item[] wobei:
                    \item $\vec{L} = \sum_i \vec{l_i}$
                    \item $\vec{S} = \sum_i \vec{s_i}$
                \end{itemize}
                \item $\Delta E_{pot} = \mu_B \cdot B \cdot M_J \cdot g_J $
                \begin{itemize}
                    \item $ g_{J}$ Landé - Faktor
                \end{itemize}
            \end{itemize}
            \begin{align*}
              S = 0 \text{ und } g_{J} = 1 &\rightarrow \text{ \textbf{normaler} Zeeman - Effekt} \\
              \text{ ansonsten } &\rightarrow \text{ \textbf{anomaler} Zeeman - Effekt}
            \end{align*}
        \end{myframe}

        \begin{myframe}{\subsecname}


        \end{myframe}

    \subsection{Auswahlregeln}

    \subsection{Lummer - Gehrcke Platte}

    \subsection{Czerny - Turner Spektrometer}
